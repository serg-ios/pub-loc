\chapter{Estado del arte}
Este capítulo hablará sobre los posibles usos de la localización en interiores, pero no sólo orientados al marketing y la publicidad, sino también a muchos otros sectores.

\section{\textit{Senion}}
En esta sección se hablará sobre una compañía llamada \textit{Senion} que ofrece múltiples aplicaciones que funcionan mediante posicionamiento en interiores, entre ellas encontramos varias orientadas a la publicidad y al marketing.

\subsection{Publicidad y marketing en interiores}
Aquí se comentarán algunos ejemplos de aplicaciones reales que han sido desarrolladas basándose en las herramientas ofrecidas por esta compañía:

\paragraph{\textit{Mall of America}.} el centro comercial más grande de Norte América ha instalado un sofisticado sistema de posicionamiento \textit{indoor} con la ayuda de la compañía \textit{Senion}. La tecnología está diseñada para mejorar la experiencia del usuario, dirigiéndolo a las tiendas, restaurantes o lugares de entretenimiento deseado a través de una \textit{app} en sus teléfonos móviles \cite{noauthor_mall_nodate}.
\paragraph{\textit{Dubai Mall}.} el centro comercial más visitado del mundo también tiene un sistema de posicionamiento \textit{indoor} que utiliza las herramienta ofrecidas por \textit{Senion}. Es lógico que un centro comercial de estas dimensiones y con unas 1200 tiendas y 200 restaurantes, necesite un sistema de localización en interiores para que sus visitantes no se pierdan. Como en el caso de uso anterior, también dispone de una \textit{app} propia \cite{noauthor_dubai_nodate}.
\paragraph{\textit{GINZA SIX}.} en este centro comercial de \textit{Tokyo} de 8 plantas, los compradores también pueden orientarse y conocer la ruta más rápida a su destino mediante un sistema de posicionamiento en interiores utilizando una red de \textit{beacons} y una aplicación para sus teléfonos móviles \cite{noauthor_ginza_nodate}.
\paragraph{Centro comercial Parque Arauco.} también tiene su propia aplicación basada en este tecnología \cite{noauthor_parque_nodate}.

\subsection{Eventos}
Otro uso del posicionamiento en interiores, sería el de facilitar los desplazamientos por festivales, eventos y otros lugares a donde la señal \textit{GPS} no puede acceder.

\paragraph{Feria de \textit{eCommerce} en Sao Paolo 2015.} En este evento, se lanzó una aplicación móvil que permitía a los visitantes conocer los \textit{stands} que estaban presentes en la feria y recibir indicaciones para llegar hasta ellos. Incluso los usuarios podían registrarse, crear un perfil y hacer su ubicación visible a otras personas, esto podría ser una posible idea para el futuro: el posicionamiento en interiores orientado a las redes sociales. Pero además, la aplicación también proporcionaba ventajas a los organizadores de la feria y propietarios de stands. Estos podían obtener datos estadísticos sobre los movimientos de los visitantes, como mapas de calor y los puntos en los que más personas suelen pararse \cite{noauthor_feria_2015}.

\subsection{Organización en las empresas}
Muchas empresas realizan estudios sobre el tiempo que pierden sus empleados yendo de un sitio para otro buscando a otro compañero, el cual quizás no se encuentre en su puesto habitual. Una aplicación que ubicase a cada empleado, podría solucionar este problema, produciendo un sustancial ahorro de tiempo para los empleados y de dinero para la empresa.

\paragraph{\textit{Ericsson optimized workplace}.} En las oficinas de \textit{Ericsson} situadas en Estocolmo, las cuales tienen una enorme superficie. La empresa \textit{Senion} en colaboración con \textit{Flowscape} \cite{noauthor_flowscape_nodate}, ha desarrollado una plataforma que incrementa en gran medida la productividad de los empleados. Esta herramienta es muy necesaria en una oficina de 20 pisos distribuidos en 4 edificios y no sólo hace muy sencillo encontrar a personas, sino también lugares o salas \cite{noauthor_ericsson_nodate}.

\section{\textit{Situm}}
En esta sección se hablará sobre la plataforma \textit{Situm}, la cual ha sido elegida para este proyecto y sus casos de éxito \cite{situm_casos_nodate}.

\paragraph{Servicios sanitarios}
Aplicación que mejora la relación entre el ciudadano y el sistema sanitario permitiendo a los pacientes y a los visitantes navegar en el interior de los hospitales a través de sus smartphones, buscar puntos de interés (consultas, salas de espera, tiendas, paradas de transporte…) y obtener rutas guiadas.
La tecnología de \textit{Situm} está presente en docenas de hospitales públicos y privados en España, Turquía y Tailandia, incluyendo algunos de los centros más grandes y complejos de Europa como el hospital Álvaro Cunqueiro, en Vigo, el hospital Lucus Augusti, en Lugo, y el hospital de Mersin \cite{situm_indoor_2018}.

\paragraph{Seguridad}
\textit{Situm} también ofrece una solución de posicionamiento en interiores, identificación y monitorización del personal de seguridad, además de nuevas e innovadores funciones basadas en la localización como alertas de hombre caído, botón de pánico y asignación de tareas geolocalizadas \cite{securitas_seguridad_espana_securitas_2018}.

\paragraph{Centros comerciales}
Esta tecnología puede ser integrada en las aplicaciones de \textit{retailers} para proveer un mejor servicio a sus clientes y para obtener información geoestadística, incrementando de esta forma el \textit{business intelligence} y mejorando el \textit{ROI} mediante el geomarketing.
\textit{Situm} ya está presente un número creciente de centros comerciales en América, Asia y Europa.

\paragraph{Servicios}
Aplicaciones de posicionamiento en interiores que permiten guiar al usuario en edificios públicos multitudinarios como aeropuertos y estaciones de tren/autobús desde su posición hasta los principales puntos de interés (\textit{check-ins}, controles de seguridad, puertas de embarque…), incluidos todos los servicios disponibles en el interior de estos edificios (aseos, tiendas, restaurantes…).
La tecnología de \textit{Situm} está siendo usada en aeropuertos internacionales, mejorando no sólo la experiencia de los viajeros, sino proveyendo también información valiosa que puede ayudar a los gestores de los edificios a monitorizar servicios clave como el personal \textit{PMR} (favoreciendo de esta forma la localización de pasajeros con movilidad reducida) y a incrementar la rentabilidad de sus operaciones no aeronáuticas mediante una mejor explotación de sus áreas comerciales.

\paragraph{Edificios corporativos}
\textit{Situm} ayuda a organizaciones públicas y privadas con sedes grandes y complejas a mejorar la eficiencia de su gestión de personal, al mismo tiempo que facilita el día a día de sus trabajadores y visitas posibilitando que puedan explorar el mapa del edificio, localizar puntos de interés (desde plazas de parking libres a salas de reuniones) y obtener rutas guiadas.
La tecnología de localización en interiores de \textit{Situm} ya está funcionando en varias sedes corporativas de Asia, Europa y Estados Unidos.

\paragraph{Industria 4.0}
\textit{Situm} provee datos en tiempo real de la interacción de activos móviles con otros elementos (como carretillas, \textit{AGVs} y demás vehículos logísticos) en el interior de ambientes industriales, proporcionando información valiosa para la optimización del proceso logístico y del uso de estos vehículos. La tecnología de \textit{Situm} destaca debido a su mínimo tiempo de despliegue, y reduce significativamente los costes de instalación gracias a que puede funcionar con la infraestructura existente.
La versatilidad de \textit{Situm} permite que su sistema pueda ser adaptado en una amplia variedad de aplicaciones para solventar problemas logísticos.
El Grupo \textit{PSA} ha escogido a \textit{Situm} como una de las mejores \textit{startups} a nivel mundial por \textit{Situm Factory Indoor}, una solución que adapta su tecnología para localizar otros elementos que no sean personas y poder hacerlo desde dispositivos dedicados.

\paragraph{Eventos}
\textit{Situm} funciona en grandes infraestructuras que acogen eventos masivos como congresos, espectáculos y encuentros deportivos (en estadios de fútbol y baloncesto). Edificios complejos con muchas señales \textit{WiFi} y \textit{Bluetooth}. Con la tecnología de \textit{Situm}, los organizadores de los eventos obtienen información valiosa que les puede ayudar a monetizar mejor sus espacios, proveyendo al mismo tiempo localización y navegación en interiores a sus visitantes.\\
Su tecnología proporciona servicios en los mayores centros de congresos, como \textit{Fira} de Barcelona (una de las instituciones feriales más importantes de Europa) a través de la solución \textit{Securitas Location} \cite{situm_indoor_2018-1}.