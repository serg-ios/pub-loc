\chapter*{Resumen}
El objetivo de este proyecto ha sido utilizar la ubicación de los teléfonos móviles como canal de comunicación entre comercios y clientes, intercambiando datos de posición por ofertas y descuentos, o por simples notificaciones acerca de productos disponibles a nuestro alrededor.

Los datos relativos a usuarios, tiendas y ofertas se almacenan en la nube. Se ha escogido la herramienta de \textit{Google Firebase} por la flexibilidad que ofrece, la sencillez con la que podemos hacer peticiones, autenticarnos, dar permisos, almacenar imágenes e incluso ejecutar código.

Se ha desarrollado una aplicación móvil para dispositivos \textit{iOS}, mediante la cual los clientes son avisados a medida que se aproximan a las tiendas con ofertas. No sólo está orientado a centros comerciales y grandes superficies, sino también a cualquier tipo de evento, feria o mercado.

Los usuarios tienen diferentes roles y permisos, de esta manera la aplicación presenta interfaces distintas para cada uno y así también se puedan delegar responsabilidades. Se ha decidido implementar una jerarquía clara: primero el administrador del sistema, seguido por los gerentes de los centros comerciales,  los propietarios de los comercios, los empleados y por último los clientes finales.

Como sistema de localización de interiores se ha empleado la tecnología de Situm, que, una vez calibrado un edificio, emplea los datos de los sensores (\textit{GPS}, \textit{Bluetooth BLE}, acelerómetros y \textit{WiFi}) para obtener la posición con un precisión unos dos metros. La precisión en \textit{iOS} es menor, ya que no se tiene acceso a la información de la \textit{WiFi}.
\section*{Palabras clave}
\textit{Situm}, Localización en interiores, \textit{Bluetooth Low Energy} o \textit{BLE}, Balizas, \textit{Firebase}, \textit{Google Maps}, Aplicación \textit{iOS}, \textit{iPhone}, \textit{Swift}.